%Academic CV LaTeX Template
% Author: Dubasi Pavan Kumar
% LinkedIn: https://www.linkedin.com/in/im-pavankumar/
% License: MIT
%
% For errors, suggestions, or improvements, please contact:
% Email: pavankumard.pg19.ma@nitp.ac.in
%

\documentclass[a4paper,11pt]{article}

% Package imports
\usepackage{latexsym}
\usepackage{xcolor}
\usepackage{float}
\usepackage{ragged2e}
\usepackage[empty]{fullpage}
\usepackage{wrapfig}
\usepackage{lipsum}
\usepackage{tabularx}
\usepackage{titlesec}
\usepackage{geometry}
\usepackage{marvosym}
\usepackage{verbatim}
\usepackage{enumitem}
\usepackage{fancyhdr}
\usepackage{multicol}
\usepackage{graphicx}
\usepackage{cfr-lm}
\usepackage{fontawesome5}

\usepackage[T2A,T1]{fontenc}
\usepackage[utf8]{inputenc}
\usepackage[azerbaijani]{babel}

% Color definitions
\definecolor{darkblue}{RGB}{0,0,139}

% Page layout
\setlength{\multicolsep}{0pt}
\pagestyle{fancy}
\fancyhf{} % clear all header and footer fields
\fancyfoot{}
\renewcommand{\headrulewidth}{0pt}
\renewcommand{\footrulewidth}{0pt}
\geometry{left=1.4cm, top=0.8cm, right=1.2cm, bottom=1cm}
\setlength{\footskip}{5pt} % Addressing fancyhdr warning

% Hyperlink setup (moved after fancyhdr to address warning)
\usepackage[hidelinks]{hyperref}
\hypersetup{
    colorlinks=true,
    linkcolor=darkblue,
    filecolor=darkblue,
    urlcolor=darkblue,
}

% Custom box settings
\usepackage[most]{tcolorbox}
\tcbset{
    frame code={},
    center title,
    left=0pt,
    right=0pt,
    top=0pt,
    bottom=0pt,
    colback=gray!20,
    colframe=white,
    width=\dimexpr\textwidth\relax,
    enlarge left by=-2mm,
    boxsep=4pt,
    arc=0pt,outer arc=0pt,
}

% URL style
\urlstyle{same}

% Text alignment
\raggedright
\setlength{\tabcolsep}{0in}

% Section formatting
\titleformat{\section}{
    \vspace{-4pt}\scshape\raggedright\large
}{}{0em}{}[\color{black}\titlerule \vspace{-7pt}]

% Custom commands
\newcommand{\resumeItem}[2]{
\item{
        \textbf{#1}{\hspace{0.5mm}#2 \vspace{-0.5mm}}
    }
}

\newcommand{\resumePOR}[3]{
    \vspace{0.5mm}
\item
    \begin{tabular*}{0.97\textwidth}[t]{l@{\extracolsep{\fill}}r}
        \textbf{#1}\hspace{0.3mm}#2 & \textit{\small{#3}}
    \end{tabular*}
    \vspace{-2mm}
}

\newcommand{\resumeSubheading}[4]{
    \vspace{0.5mm}
\item
    \begin{tabular*}{0.98\textwidth}[t]{l@{\extracolsep{\fill}}r}
        \textbf{#1} & \textit{\footnotesize{#4}} \\
        \textit{\footnotesize{#3}} &  \footnotesize{#2}\\
    \end{tabular*}
    \vspace{-2.4mm}
}

\newcommand{\resumeProject}[4]{
    \vspace{0.5mm}
\item
    \begin{tabular*}{0.98\textwidth}[t]{l@{\extracolsep{\fill}}r}
        \textbf{#1} & \textit{\footnotesize{#3}} \\
        \footnotesize{\textit{#2}} & \footnotesize{#4}
    \end{tabular*}
    \vspace{-2.4mm}
}

\newcommand{\resumeSubItem}[2]{\resumeItem{#1}{#2}\vspace{-4pt}}

\renewcommand{\labelitemi}{$\vcenter{\hbox{\tiny$\bullet$}}$}
\renewcommand{\labelitemii}{$\vcenter{\hbox{\tiny$\circ$}}$}

\newcommand{\resumeSubHeadingListStart}{
\begin{itemize}[leftmargin=*,labelsep=1mm]}
        \newcommand{\resumeHeadingSkillStart}{
        \begin{itemize}[leftmargin=*,itemsep=1.7mm, rightmargin=2ex]}
                \newcommand{\resumeItemListStart}{
                \begin{itemize}[leftmargin=*,labelsep=1mm,itemsep=0.5mm]}

                        \newcommand{\resumeSubHeadingListEnd}{
                    \end{itemize}\vspace{2mm}}
                \newcommand{\resumeHeadingSkillEnd}{
            \end{itemize}\vspace{-2mm}}
        \newcommand{\resumeItemListEnd}{
    \end{itemize}\vspace{-2mm}}
\newcommand{\cvsection}[1]{%
    \vspace{2mm}
    \begin{tcolorbox}
        \textbf{\large #1}
    \end{tcolorbox}
    \vspace{-4mm}
}

\newcolumntype{L}{>{\raggedright\arraybackslash}X}%
\newcolumntype{R}{>{\raggedleft\arraybackslash}X}%
\newcolumntype{C}{>{\centering\arraybackslash}X}%

% Commands for icon sizing and positioning
\newcommand{\socialicon}[1]{\raisebox{-0.05em}{\resizebox{!}{1em}{#1}}}
\newcommand{\ieeeicon}[1]{\raisebox{-0.3em}{\resizebox{!}{1.3em}{#1}}}

% Font options
\newcommand{\headerfonti}{\fontfamily{phv}\selectfont} % Helvetica-like (similar to Arial/Calibri)
\newcommand{\headerfontii}{\fontfamily{ptm}\selectfont} % Times-like (similar to Times New Roman)
\newcommand{\headerfontiii}{\fontfamily{ppl}\selectfont} % Palatino (elegant serif)
\newcommand{\headerfontiv}{\fontfamily{pbk}\selectfont} % Bookman (readable serif)
\newcommand{\headerfontv}{\fontfamily{pag}\selectfont} % Avant Garde-like (similar to Trebuchet MS)
\newcommand{\headerfontvi}{\fontfamily{cmss}\selectfont} % Computer Modern Sans Serif
\newcommand{\headerfontvii}{\fontfamily{qhv}\selectfont} % Quasi-Helvetica (another Arial/Calibri alternative)
\newcommand{\headerfontviii}{\fontfamily{qpl}\selectfont} % Quasi-Palatino (another elegant serif option)
\newcommand{\headerfontix}{\fontfamily{qtm}\selectfont} % Quasi-Times (another Times New Roman alternative)
\newcommand{\headerfontx}{\fontfamily{bch}\selectfont} % Charter (clean serif font)

\begin{document}
\headerfontiii

% Header
\begin{center}
    {\Huge\textbf{Yusif Mehdiyev}}
\end{center}
\vspace{-5mm}

\begin{center}
    \small{
        +994555418363 | \href{mailto:yusifmehdiyev55@gmail.com}{yusifmehdiyev55@gmail.com}|
        \href{mailto:ymehdiyev22216@ada.edu.az}{ymehdiyev22216@ada.edu.az}|
        \href{https://josephmehdiyev.org/}{josephmehdiyev.org}
    }
\end{center}
\vspace{-5mm}

\begin{center}
    \small{
        \socialicon{\faGithub} \href{https://github.com/JosephMehdiyev}{JosephMehdiyev}
        \socialicon{\faLinkedin} \href{https://www.linkedin.com/in/joseph-mehdiyev-6155982ab/} {Joseph Mehdiyev}
    }
\end{center}
\vspace{-5mm}
\begin{center}
    \small{Azərbaycan, Bakı}
\end{center}

\section{\textbf{Təhsil}}
\resumeSubHeadingListStart

\resumeSubheading
{ADA Universiteti}{Bakı, Azərbaycan}
{Riyaziyyat üzrə Bakalavr dərəcəsi}{2024 - Hal-hazırda}
\resumeItemListStart
\item GPA: 4.0/4.0
\item Rektorun Şərəf Siyahısı
\item Lotfi A. Zadeh Şərəf Siyahısı
\item IELTS: 7.5 (8.5/8.5/6.5/6)
\resumeItemListEnd

\resumeSubheading
{12 nömrəli Tam Orta Məktəb}{Bakı, Azərbaycan}
{Ümumi Orta Təhsil}{2013 - 2024}
\resumeItemListStart
\item CGPA: 4.97/5.00
\item SAT: 1500/1600 (790 Riyaziyyat,710 Oxuma/Yazma)
\item 11-ci sinif DIM Buraxılış İmtahanı: 282/300
\item 9-cu sinif Buraxılış İmtahan: 263/300
\item 9-cu sinif Fərqlənmə Attestatı
\resumeItemListEnd
\resumeSubHeadingListEnd

\section{\textbf{Təcrübə}}
\resumeSubHeadingListStart
\resumeSubheading
{ADA Universitetində Professor Köməkçisi}{Bakı, Azərbaycan}
{Xətti Cəbr fənnindən Professor Köməkçisi}{Fevral 2025 - Hal-hazırda}
\resumeItemListStart
\item Ev tapşırıqlarının və imtahanların qiymətləndirilməsi, imtahan nəzarəti, ofis saatlarında tələbələrə dəstək və dərslə əlaqəli
inzibati işlər görürəm.
\resumeItemListEnd
\resumeSubHeadingListEnd

\section{\textbf{Layihələr}}
\resumeSubHeadingListStart
\resumeProject
{Cormat: Riyazi funksiyalarının 3D vizuallaşdırılması üçün proqram}
{Alətlər: C++, Cmake, OpenGl, GLFW, ImGui, GLSL shaders}
{Mart 2024 - Hal-hazırda}
{{}\href{https://github.com/JosephMehdiyev/cormat}{\textcolor{darkblue}{\faGithub}}}
\resumeItemListStart
\item C++ və OpenGL istifadə edərək sıfırdan 3D riyazi vizuallaşdırma aləti yaradılıb.
\item Xətti cəbr çevrilmələrini (məsələn, köçürmə, fırlanma, miqyaslama) proqramlaşdırma yolu ilə həyata keçirilib.
\item Layihəni OOP prinsipləri əsasında modullu və genişlənə bilən şəkildə qurulub.
\item Layihəni təşkil etmək üçün CMake-dən istifadə edilib.
\resumeItemListEnd
\resumeSubHeadingListEnd

\section{\textbf{Nəşrlər}}
\resumeSubHeadingListStart
\resumeProject
{Ehtimal və Statistika qeydləri}
{Şəxsi istifadə üçün kitab stilində qeydlər}
{Mart 2024 - Hal-hazırda}
{{}\href{https://github.com/JosephMehdiyev/Statistics-and-Probability-with-Code-Applications/blob/main/main.pdf}{\textcolor{darkblue}{\faGithub}}}
\resumeItemListStart
\item Ehtimal nəzəriyyəsi, statistik çıxarış və statistik modellərə dair hərtərəfli qeydlər
\item Ehtimal paylanmaları, hipotez testləri, reqressiya analizi kimi əsas anlayışları öyrənildi
\resumeItemListEnd

\resumeProject
{Real Analiz qeydləri}
{Şəxsi istifadə üçün kitab stilində qeydlər}
{Sentyabr 2024 - Hal-hazırda}
{{}\href{https://github.com/JosephMehdiyev/MathmeticalAnalysisNotes/blob/main/main.pdf}{\textcolor{darkblue}{\faGithub}}}
\resumeItemListStart
\item Real Analizin əsas mövzularını əhatə edən ətraflı qeydlər
\item Mürəkkəb riyazi anlayışların başa düşülməsi üçün sübut texnikaları ilə məşğul olundu
\resumeItemListEnd
\resumeSubHeadingListEnd

\section{\textbf{Bacarıqlar}}
\resumeHeadingSkillStart
\resumeSubItem{Riyazi Biliklər:}
{Vektor Analizi, Xətti Cəbr, Qrafik Nəzəriyyəsinin əsasları, Ədədlər Nəzəriyyəsi, Ehtimal və Statistika, Real Analiz, Qrup Nəzəriyyəsi}
\resumeSubItem{Proqramlaşdırma Dilləri:}
{ C, C++, Python, R, Lua}
\resumeSubItem{Verilənlər Bazası Sistemləri:}
{ PostgreSQL}
\resumeSubItem{DevOps \& Versiya Kontrolü:}
{ Git, Cmake, Docker }
\resumeSubItem{Dil Bilikləri:}
{Azərbaycan dili (Ana dili), Türk dili (C1), İngilis dili (C1; IELTS 7.5), Fransız dili (B1)}
\resumeSubItem{Digər Alətlər \& Texnologiyalar:}
{ Latex, Linux (arch və debian əsaslı), Microsoft Excel, RStudio, Vim/Neovim, Tableu}
\resumeHeadingSkillEnd

\section{\textbf{Yarışmalar, Mükafatlar və Fəaliyyətlər}}
\resumeSubHeadingListStart
\resumeProject
{CodeForces Proqramlaşdırma Yarışları}
{Rəqabətli proqramlaşdırma bacarıqlarının təkmilləşdirilməsi}
{May 2024 - Hal-hazırda}
{{}\href{https://codeforces.com/profile/JosephMehdiyev}{\textcolor{darkblue}{\faIcon{globe}}}}
\resumeItemListStart
\item Alqoritmik anlayışların və məlumat strukturlarının mənimsənilməsi (DP, qraf alqoritmləri, ağaclar)
\item Problemlərin sürətli həlli və optimal kod yazmaq üçün təcrübələr
\resumeItemListEnd

\resumeProject
{Azərbaycan Olimpiadları bloqqeri}
{Art of Problem Solving Forum}
{Aprel 2022 - Sentyabr 2022}
{{}\href{https://artofproblemsolving.com/community/user/948405}{\textcolor{darkblue}{\faIcon{globe}}}}
\resumeItemListStart
\item Azərbaycanın milli və beynəlxalq olimpiadaları üçün TSTST, TST və digər problemlərin arxivləşdirilməsi
\resumeItemListEnd

\resumeProject
{Azərbaycan Gənclər Riyaziyyat Düşərgəsi İştirakçısı}
{Azərbaycan olimpiad düşərgəsi (IMO, BMO, JBMO üçün)}
{Sentyabr 2021 - İyun 2022}
{}
\resumeItemListStart
\item Komanda yoldaşları ilə çətin riyazi problemlərin həllində əməkdaşlıq etdim.
\item Tənqidi düşüncə və problem həll strategiyalarının təkmilləşdirdim.
\resumeItemListEnd

\resumeProject
{Azərbaycan Riyaziyyat Komandası Seçim İmtahanları İştirakçısı}
{Azərbaycan TST (IMO, BMO, JBMO üçün)}
{Sentyabr 2021 - İyun 2022}
{}
\resumeItemListStart
\item Azərbaycan milli komandasının seçim imtahanlarında iştirak etdim.
\resumeItemListEnd
\resumeSubHeadingListEnd

% [REFERENCES SECTION REMAINS EXACTLY THE SAME]

\end{document}
