\documentclass[12pt]{article}
\usepackage[a4paper,margin=1in]{geometry}
\usepackage{hyperref}
\usepackage{enumitem}
\usepackage{lmodern}
\usepackage{titlesec}
\usepackage{xcolor}
\usepackage{tabularx}
\usepackage{amssymb}
\usepackage{helvet}
\setlength{\parindent}{0pt}
\hypersetup{
    colorlinks=true,
    linkcolor=blue,
    urlcolor=blue,
citecolor=blue,    }
\titleformat{\subsection}{\bfseries}{}{0em}{}
\newcommand{\vocab}[1]{\textbf{\color{violet}#1}}
\providecommand{\alert}{\vocab}

\title{Transcript of Coursework}
\author{Yusif (Joseph) Mehdiyev}
\date{Last updated October 3, 2025}

\begin{document}

\maketitle

This file contains information  about the important courseworks and the self-study subjects  I had over in my career. The structure of the document is self-explanatory, The letter(s) in the left of the course code shows the final grade, if exists. If you have any questions regarding myself or want to see the official documents, be free to e-mail me!
\renewcommand{\contentsname}{Contents:} % Remove "Contents" heading
\tableofcontents
\phantomsection
\section*{\textcolor{blue}{\ \ \ \ \  University of Glasgow}}
\addcontentsline{toc}{section}{University of Glasgow}
\begin{itemize}[leftmargin = *]
     \item[]
        \begin{tabularx}{\textwidth}{@{}lX@{}}
            ? & \textbf{MATHS1021}, \textit{Foundation of Mathematics}, Fall 2025, Anna Puskas (main head)
            \\
            &
            \href{https://www.gla.ac.uk/coursecatalogue/course/?code=MATHS1021}{Course Link} \\
            &
            Foundations of Mathematics aims to transition students to university level mathematics through development of abstract structures and reasoning skills, the interplay between algebra and geometry, and to ensure students have a strong command of the basics of mathematics that is crucial to our degree programmes. A strong focus throughout the course will be placed on developing mathematical communication skills.         \\
            & \textit{Textbook:} 
            M. Liebeck, A Concise Introduction to Pure Mathematics, Routledge, 4th ed. (2015);
            J. Poole, Linear Algebra: A Modern Introduction, Brooks Cole, 4th ed. (2014); 
            J. Stewart, Calculus, Brooks Cole, 9th ed., International Metric Version (2020).

        \end{tabularx}

    \item[]
        \begin{tabularx}{\textwidth}{@{}lX@{}}
            ? & \textbf{MATHS2001}, \textit{Multivariable Calculus}, Fall 2025, Ana G. Lecuona
            \\
            &
            \href{https://www.gla.ac.uk/coursecatalogue/course/?code=MATHS2001}{Course Link} \\
            &
            This course on multivariate calculus gives a practical introduction to differentiating and integrating in multiple dimensions, and to fundamental concepts found in diverse fields such as geometry and physics. It is an essential course for intending honours students. The emphasis is on methods and applications.            \\
            & \textit{Textbook:} Multivariable Calculus, Metric Edition (9th Edition) by James Stewart; ISBN: 9780357113509.
        \end{tabularx}
            \item[]
        \begin{tabularx}{\textwidth}{@{}lX@{}}
            ? & \textbf{MATHS2004}, \textit{Linear Algebra}, Fall 2025,  Christian Korff
            \\
            &
            \href{https://www.gla.ac.uk/coursecatalogue/course/?code=MATHS2004}{Course Link} \\
            &
            This course covers the fundamentals of linear algebra that are applicable throughout science and engineering, and in particular in the physical, chemical and biological sciences, statistics and other parts of mathematics. It is an essential course for intending honours students. The emphasis is on methods and applications.            \\
            & \textit{Textbook:} Lecture notes mainly used, suplementary book is Nicholson, Linear Algebra with Applications (2023 A-D edition).
        \end{tabularx}
         \item[]
        \begin{tabularx}{\textwidth}{@{}lX@{}}
            ? & \textbf{MATHS2032}, \textit{Introduction to Real Analysis}, Fall 2025,  Mark Powell
            \\
            &
            \href{https://www.gla.ac.uk/coursecatalogue/course/?code=MATHS2032}{Course Link} \\
            &
            This course is a first introduction to real analysis. The common thread running through the course is the notion of limit. The precise definition of this notion will be given for both sequences and series. It is an essential course for intending honours students. The emphasis is on developing and applying standard techniques of proof to give rigorous arguments from basic definitions.            \\
            & \textit{Textbook:}  None, Professor taught the concepts with a chalk and board.
        \end{tabularx}
        
        \begin{tabularx}{\textwidth}{@{}lX@{}}
            ? & \textbf{STATS2002}, \textit{Probability}, Fall 2025,   Alexey Lindo
            \\
            &
            \href{https://www.gla.ac.uk/coursecatalogue/course/?code=STATS2002}{Course Link} \\
            &
This course introduces students to fundamental concepts in univariate probability theory.            \\
            & \textit{Textbook:}  None, Lecture Notes.
        \end{tabularx}

        \begin{tabularx}{\textwidth}{@{}lX@{}}
             ? & \textbf{STATS2003}, \textit{Statistical Methods, Models and Computing 1}, Fall 2025, Iain Bell and Mitchum Bock
            \\
            &
            \href{https://www.gla.ac.uk/coursecatalogue/course/?code=STATS2003}{Course Link} \\
            &
           This course introduces students to key concepts in the statistical sciences including data visualisation, parameter estimation, statistical inference and analysis using statistical software.\\
            & \textit{Textbook:}  Probability and Statistics with R by Ugarte, Militino and Arnholt.
        \end{tabularx}


\end{itemize}

\phantomsection
\section*{\textcolor{blue}{\ \ \ \ \  ADA University}}
\addcontentsline{toc}{section}{ADA University}
\begin{itemize}[leftmargin = *]
    \item[]
        \begin{tabularx}{\textwidth}{@{}lX@{}}
            A & \textbf{MATH 1111}, \textit{Calculus I}, Spring 2025, Javanshir Azizov
            \\
            & Standard Calculus Course. Limits, Convergence and Divergence of Limits, Differentiation of the functions, Anti-Derivatives,
            Integration, Fundamental Theorem of Calculus, Area Between Curves, Volumes by Rotation, Initial Value Problems, Introductory
            Differential Equations.
            \\
            & \textit{Textbook:} Thomas’ Calculus Early Transcendentals.
        \end{tabularx}
    \item[]
        \begin{tabularx}{\textwidth}{@{}lX@{}}
            A & \textbf{MATH 1201}, \textit{Abstract Algebra}, Spring 2025, Rafael Alizade
            \\
            & Group Theory, Permutations, Cosets and Theorem of Lagrange, Rings, Fields, Integral Domains, Rings of Polynomials, Vector
            Spaces, Euler's and Fermat's Theorems.
            \\
            & \textit{Textbook:} John Fralelgh's A First Course in Abstract Algebra.
        \end{tabularx}
    \item[]
        \begin{tabularx}{\textwidth}{@{}lX@{}}
            A & \textbf{CSCI 1101}, \textit{Programming Principles I}, Spring 2025, Rashad Aliyev
            \\
            & Standard Introductory Programming course for C and C++. Logic, Variables, Loops, Arrays, Strings, Functions, Pointers,
            Dynamic Memory Allocation, Bitwise Operation, Elementary Algorithms.
            \\
            & \textit{Textbook:} None.
        \end{tabularx}
    \item[]
        \begin{tabularx}{\textwidth}{@{}lX@{}}
            A & \textbf{MATH 3501}, \textit{Linear Algebra}, Fall 2025, Elchin Hasanilzade
            \\
            & Computation focused Linear Algebra course. Vectors, Hyperplanes, System of Linear Equations, Matrices, Echalon and Row
            Canonical Forms, Span and Basis, Permutations, Determinants, Inverse of Matrices, Linear mappings and transformations, Change of Basis,
            Similarity, Orthogonal Basis, Gram-Schmit Algorithm, Diagonalization, Eigenvalues, Eigenvectors, Linear Functional and Dual Spaces.
            \\
            & \textit{Textbook:} Schaum's Outline of Linear Algebra.
        \end{tabularx}
    \item[]
        \begin{tabularx}{\textwidth}{@{}lX@{}}
            A & \textbf{MATH 1100}, \textit{Pre-Calculus}, Fall 2025, Javanshir Azizov
            \\
            & Standard (compulsory!) Pre-Calculus course. Definition of Functions, Injective and Surjective Functions, Trigonometric Functions etc.
            \\
            & \textit{Textbook:} Stewart's Precalculus.
        \end{tabularx}
    \item[]
        \begin{tabularx}{\textwidth}{@{}lX@{}}
            A & \textbf{SITE 1101}, \textit{Principles of Information Systems}, Fall 2025, Rashad Aliyev, Araz Yusubov
            \\
            & A course that introduces concepts of general Information Systems. Introduction to Programming, Computer Networks, Principles
            of Designing Software, Development Pipeline etc.
            \\
            & \textit{Textbook:} None.
        \end{tabularx}
    \item[]
        \begin{tabularx}{\textwidth}{@{}lX@{}}
            P & \textbf{PDEV 2302}, \textit{Data And Computing Skills}, Fall 2025, Khalil Israfilzada.

            \\
            & Introductory course focused on Excel and Data Analysis.
            \\
            & \textit{Textbook:} None.
        \end{tabularx}
\end{itemize}
\phantomsection
\section*{\textcolor{blue}{\ \ \ \ \  Self-Study}}
\addcontentsline{toc}{section}{Self-Study}
\begin{itemize}
    \item[] \textbf{Probability and Statistics}
        \\
        Discrete and Continous Random Variables, Expectation, Variance, Distribution of Random Variables, Cumultative Distribution
        Function, Probability Density Function, Convergence in Distribution and Probability, Law of Large Numbers, Central Limit Theorem,
        Statistical Models, Parametric and non-Parametric Inference, Point Estimator, Bias, Method of Moments, Maximum Likelihood,
        Bootsrap, Hypothesis Testing and p-values, Risk and Loss Functions etc.
        \\
        \textit{Textbook: Wasserman's All of Statistics  A Concise Course in Statistical Inference}
        \\
        \textit{Technologies Practiced:} Python Libraries such as Pandas, Numpy, Scipy, Matplotlib, Seaborn.
        \\
        \textit{Status: Have studied Part I and Part II of the book.}
    \item[] \textbf{Real Analysis}
        \\
        Axiom of Completeness, Supremum and Infimums, Converence of Sequences, Monotone Convergence Theorem, Subsequences and
        Bolzano-Weierstrass Theorem, Cauchy Sequences, Topolofy of $\mathbb{R}$: Open, Closed, Perfect, Compact Sets, Functional Limits,
        Continous and Uniformly Continous Functions, Intermediate Value Theorem, Derivatives, Mean Value Theorem
        \\
        \textit{Textbook: Stephen Abbot's Understand Analysis}
        \\
        \textit{Status: Have Studied all the chapters except for Integrals and Series chapters}
    \item[] \textbf{(Analytical) Classical Mechanics, Physics}
        \\
        Currently in Chapter I, started to study for my Physics Simulation Projects.
        \\
        \textit{Textbook: Herbert Goldstein's Classical Mechanics}

\end{itemize}

\end{document}
